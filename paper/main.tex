\documentclass[letterpaper, 12 pt, conference]{ieeeconf} 
\IEEEoverridecommandlockouts                              
\overrideIEEEmargins
\usepackage{setspace}
\usepackage{hyperref}
\hypersetup{
    colorlinks=true,
    urlcolor=blue,
    }

\urlstyle{same}
\usepackage{wrapfig}




% The following packages can be found on http:\\www.ctan.org
\usepackage{graphics} % for pdf, bitmapped graphics files
%\usepackage{epsfig} % for postscript graphics files
%\usepackage{mathptmx} % assumes new font selection scheme installed
\usepackage{times} % assumes new font selection scheme installed
%\usepackage{amsmath} % assumes amsmath package installed
%\usepackage{amssymb}  % assumes amsmath package installed
\usepackage{graphicx}

\begin{document}

\thispagestyle{empty}
\pagestyle{empty}

\begin{titlepage}
    \begin{center}
        \vspace*{2.5cm}
 
        \textbf{Feedback Dashboarding Tool}
 
        \vspace{0.5cm}
         Thesis Subtitle
 
         \vfill
        
 
        \textbf{Genevieve Couture}
        200426387 grc839@uregina.ca
 
        \textbf{Payton Gilbertson}
        200248472 gilbertp@uregina.ca
 
        \textbf{Liam Gulak}
        200393578 lgq292@uregina.ca
 
        \textbf{Yunseok Kim}
        200434122 yjk172@uregina.ca
 
        \textbf{Zachary Huber}
        200408160 zeh434@uregina.ca
 
        \vspace{1.5cm}  
        CS 476 Project Report
             
        \vspace{0.8cm}
      
             
        University of Regina\\
        March 21, 2023
             
    \end{center}
 \end{titlepage}


%%%%%%%%%%%%%%%%%%% BODY %%%%%%%%%%%%%%%%%%%%%%%%%%%%%%%%%%%%%%%%%%%%%%%%%%%%%%%%


\doublespacing 
\onecolumn

\section{Problem Definition}
\subsection{Outline the problem requirements and include the application domain and motivations of your project (1 page).}
\linebreak
Business and product owners want meaningful feedback from their customers to assist in their growth and success. It can be difficult to collect this information without onsite hardware and processing the data provided can be a timely task. As with many forms of customer engagement and ratings, customers can be unwilling to participate in traditional methods of feedback if too many steps are required, so the data collected will not be an accurate portrayal of opinions.
\newline


In order to solve this problem, we have created a web application that allows for customers to provide their feedback in a simplistic way, and displays this information for businesses or owners in a way that it simple to understand without the need for onsite hardware. By allowing the customer to interact with a site that is not cumbersome, and can be done from the convenience of their smart phone, many more customers will be willing to submit their thoughts and opinions.
\newline

Our web application is created specifically to benefit business owners or managers who want to monitor the opinions of their clientele in a simplistic way. Based on our previous experiences, these surveys for clientele are often annoying to complete or difficult to find. Often they require typing in a website manually from the back of a receipt which is not worth most people's time. By creating a website that consolidates everything to a simple QR code that can be scanned by a smart phone either posted at the physical location to be rated, or printed on a receipt to be done after the fact. 
\newline

By making surveys more accessible and simple to fill out, more people will be willing to participate, making survey results more accurate. As of now, most people who leave reviews or fill out surveys have extreme opinions which may sway the results in one way or another, so by creating this application, we have hopes that more people will share their opinions making the results much more well rounded.

%%%%%%%%%%%%%%%%%%%%%%%%%%%%%%%%%%%%%%%%%%%%%%%%%%%%%

\newpage    
\section{Application Benefits}
\subsection{What are the benefits of your application when compared to existing systems. Choose two systems only and include their references (2 pages).}
\linebreak
Our application is a simple tool for conducting self-administered surveys. Our users can log into their profiles to view customer feedback to improve their businesses. QR codes can be generated for wide distribution and quick access. SurveyMonkey (\url{https://www.surveymonkey.com/}) and QuestionPro (\url{https://www.questionpro.com/}) are comparable existing systems. Some benefits of our application are that the users can received unlimited number of responses for their surveys without cost and all features to create the surveys are free as well. Our application does not have many templates and complicated tools that may negatively impact loading time. Respondents do not need to create an account to respond to surveys. 

\hfill 

\subsubsection*{Survey Monkey}\hfill 

Survey Monkey is one of the more popular survey platforms as of now. It has lots of different options for different groups of people who may be interested in gaining an insight into people's opinions. With its wide array of available options, it is what many people think of when someone mentions a survey website.
\newline

Survey Monkey's website is very overwhelming to look at on first glance, but after a bit of digging you are able to find a bit of information on the packages available. In the past, they have had a pretty large amount of features for free users, but now their free package is very limited. Currently, their free package only allows for unlimited surveys, but with a restriction of 10 questions each, and only allows for a limited amount of responses. This means that if you want to get information with a large group of people, you would have to create multiples of the same surveys and compile the data together manually.
\newline

Overall, SurveyMonkey is a good tool for users who want to create surveys quickly and easily, but it may not be the best fit for people who want more expansive options for length of survey, and ability to view feedback. The limited free plan and cost may also be a reason for people to choose an alternative.

\break
\subsubsection*{QuestionPro}\hfill

QuestionPro, similar to Survey Monkey, has many different options depending on the customer and type of data they want to collect. It has many of the same options as Survey Monkey has when it comes to questions type, but seems like it's list of features might be more expansive.
\newline

QuestionPro has a very similar home page to Survey Monkey, but made it easier to find the differences between their packages available. This site has many more features available for free users than Survey Monkey, but still has a limit to the amount of questions and responses unless you pay. Rather than limiting to 10 question per survey, QuestionPro allows free users to have up to 100 questions per survey with up to 300 responses which is a very large increase in comparison to Survey Monkey.
\newline

QuestionPro is definitely a powerful tool for creating surveys online, but may be confusing for users with limited technology experiences. The cost for full access is also a restriction for some, but in comparison to Survey Monkey, the free version includes a much larger amount of features and usability.
\newline

Both of these other platforms for similar survey creation to ours lock many features behind a paywall. Our goal to create a very simple survey platform eliminates the confusion from having too many options or limited response allowances to make it more user friendly. We also only have a free option available for all users, so no features are hidden behind a paywall, allowing everyone to have the same experience while using it.

%%%%%%%%%%%%%%%%%%%%%%%%%%%%%%%%%%%%%%%%%%%%%%%%%%%%%

\newpage
\section{Requirements Elicitation and Specification}
\subsection{Functional requirements list (only the ones that you have implemented) for each user role (two exactly). Name each requirement and explain it briefly.}
\linebreak

\textbf{Feedback Gatherer}
\begin{enumerate}
   \item \textbf{Register:} Feedback gatherer creates an account by entering the required information. 
   \item \textbf{Create Survey:} Feedback gatherer creates a survey by entering questions and selecting type of responses.
   \item \textbf{View QR Code:} Feedback gatherer views the generated QR code for the survey.
   \item \textbf{Delete Survey:} Feedback gatherer deletes a survey that is no longer required.
   \item \textbf{View Statistics and Responses:} Feedback gatherer views statistics and responses of a selected survey.
\end{enumerate}

\newline 
\hfill \break

\textbf{Respondent}
\begin{enumerate}
   \item \textbf{Scan QR Code:} Respondent scans the QR code to view the survey.
   \item \textbf{Enter Responses:} Respondent enters responses to the survey questions.
   \item \textbf{Submit Response:} Respondent submits responses to the survey questions.
\end{enumerate}

\newpage

\subsection{For each user role, provide the use case diagram with all the use cases and actors.}
\hfill \break

\includegraphics[width=0.70\textwidth]{caseDiagram.png}

\newpage
\subsection{Describe in detail two use cases using the activity diagram. Choose the most complex use cases.}
\begin{enumerate}
    \item The feedback gatherer can create a survey after registering. The feedback gatherer has the option to delete an existing survey or view statistics and responses to an existing survey. After creating a survey, the feedback gatherer can view the QR code to administer the created survey.
    \hfill \break

    \includegraphics[width=0.70\textwidth]{feedbackGatherer.png}
\newpage
    \item The respondent can scan the QR code to take a survey created by a feedback gatherer. The respondent enters responses to the survey and submit the responses.
    \linebreak
    \hfill \break
    \includegraphics[width=0.70\textwidth]{respondent.png}
    

\end{enumerate}
\hfill \break
\newpage

\subsection{Software qualities: Correctness, Time-efficiency and Robustness. Include at least two concrete examples for each quality for each user role. The sign-in and sign-up use cases are excluded from the examples.}
\linebreak
    \hfill \break
\textbf{Correctness:}
\begin{enumerate}
    \item[] Feedback Gatherer:
    \begin{itemize}
        \item A survey correctly displays the questions and answer choices entered by the feedback gatherer during creation. 
        \item Accurate statistics for a survey are shown to the feedback gatherer.
    \end{itemize}
    \item[] Respondent:
    \begin{itemize}
        \item Scanning the QR code takes the respondent to the correct survey.
        \item Responses to a survey are saved correctly.
    \end{itemize}
\end{enumerate}
\hfill \break

\textbf{Time Efficiency:}
\begin{enumerate}
    \item[] Feedback Gatherer:
    \begin{itemize}
        \item  When a feedback gatherer creates the survey, the QR code for the survey is created in a timely manner.
        \item When responses to a survey are submitted, they are available for view by the feedback gatherer in a timely manner.
    \end{itemize}
    \item[] Respondent:
    \begin{itemize}
        \item A respondent is taken to the survey in a timely manner after scanning a QR code for a survey.
        \item A respondent is given a confirmation in a timely manner after submitting a survey.
    \end{itemize}
\end{enumerate}

\hfill \break
\newpage

\textbf{Robustness:}
\begin{enumerate}
    \item[] Feedback Gatherer:
    \begin{itemize}
        \item  If a feedback gatherer attempts to submit a blank survey, submission is denied with a warning.
        \item If a feedback gatherer forgets to select a type of answer to a question, submission is denied with a warning.
    \end{itemize}
    \item[] Respondent:
    \begin{itemize}
        \item If a respondent attempts to submit a survey with one or more unanswered questions, submission is denied and unanswered questions are marked.
        \item If a respondent attempts to exit the survey after answering one or more questions, a warning is shown.
    \end{itemize}
\end{enumerate}

%%%%%%%%%%%%%%%%%%%%%%%%%%%%%%%%%%%%%%%%%%%%%%%%%%%%%
\newpage 
\section{Top-level and Low-level Software Design}
\subsection{Provide the MVC architecture according to the selected Web frame- work. Also, describe at least three benefits of MVC for your application.}
\newline
\hfill

Django uses MVT (Model View Template) architecture which is a variation of the traditional MVC (Model View Controller) architecture. Similar to the MVC, the Model in MVT represents the data and the business logic. The View in MVT is split into two parts. The Template displays the data while the View handles the user input and interacts with the Model to retrieve and manipulate the data. Similar to the MVC, MVT separates layers and keeps the code organized and maintainable while also simplifying the View by separating it into two parts. 
\newline

The MVT architecture allows the developer to work on each layer in parallel and then integrate them efficiently. In our project, we used this feature to build and test the three main functions: user authentication process, creation of surveys by the feedback gatherers, and submission of surveys by the respondents. 
\newline

For the user registration and authentication processes, we first created the template for the user to register and log in, and the table in the model to store the user information data. Views were then set up to handle the user input and to retrieve and manipulate the data. Finally, the template was finalized to display the data.
\newpage
    
\subsection{Observer and Factory design patterns. Explain in detail the usability of these two patterns for your specific application. Include the complete class diagram for each pattern. Also provide the algorithms corresponding to the pattern’s important methods. For each class, provide the data types of the attributes and prototypes of the methods.}
\linebreak
    \hfill \break
    
\subsection{Provide the class diagram of the whole system by incorporating the two design patterns.}
\linebreak
    \hfill \break
%%%%%%%%%%%%%%%%%%%%%%%%%%%%%%%%%%%%%%%%%%%%%%%%%%%%%

\section{Software Construction}
\subsection{Submit the screenshot of the entire structure of the code within the web framework.}
\linebreak
    \hfill \break
    
\subsection{Deployment diagram regarding the hardware configuration of the code. Indicate the supported Web browsers, the application/Web servers and the database solution.}
\linebreak
    \hfill \break
    
\subsection{Screenshots of all table contents of the system data.}
\linebreak
    \hfill \break
    
\subsection{GitHub link of the entire program. All students must contribute equally to the programming part. The commit log of each student will be checked within GitHub. Website builders, such as Wordpress, are not allowed.}
\linebreak
    \hfill \break
\url{https://github.com/zachary-huber/feedback-app}
\hfill \break

\subsection{Link of your Web-based application. The application should be accessible online and runnable.}
\linebreak
    \hfill \break
\url{https://www.viewnote.app/}
\hfill \break


%%%%%%%%%%%%%%%%%%%%%%%%%%%%%%%%%%%%%%%%%%%%%%%%%%%%%

\section{Technical Documentation}
\subsection{List of programming languages.}
\begin{enumerate}
   \item Javascript
   \item HTML
   \item CSS
   \item PHP
\end{enumerate}

\subsection{List of reused algorithms and programs. Include their sources.}
\begin{enumerate}
   \item  Django web and database frameworks:
   \begin{description}
     \item \url{https://docs.djangoproject.com/en/4.1/}
   \end{description}
   \item thing
\end{enumerate}

\subsection{List of software tools and environments. Provide briefly their benefits specifically for your application.}
\begin{enumerate}
   \item Django
   \begin{description}
     \item The main functionality of our web application requires a web framework with a database that can be used to store and load user’s data such as their surveys and responses. Django allows quick and efficient development of our web application by providing the framework and the modules. The nuisance of creating and connecting to a database is already implemented in Django. We were able to quickly test the storage and loading of our data in the early stages of development. Migration of our data was also made simple by Django because it only requires running a few commands in the terminal. 
   \end{description}
   \item ngrok
   \item mySQL
\end{enumerate}

%%%%%%%%%%%%%%%%%%%%%%%%%%%%%%%%%%%%%%%%%%%%%%%%%%%%%

\section{Acceptance Testing: select test cases for both user roles. The sign-in and sign-up are excluded from testing.}
\subsection{Correctness testing using four test cases (screenshots of both inputs and outputs).}
\newline
\hfill

\textbf{Feedback gatherer:}
\begin{itemize}
    \item[] Input: Feedback gatherer creates a survey by selecting the types of answers and entering the questions. 
    \item[] Output: When they click the “Create Survey” button, a QR code is generated.
    \item[] Input: Feedback gatherer scans the QR code.
    \item[] Output: Feedback gatherer is taken to the correct survey displaying the correct questions.
    \item[] Input: Feedback gatherer clicks on one of the past surveys.
    \item[] Output: Feedback gatherer can view the responses.
    \item[] Input: Feedback gatherer clicks the delete button on one of the past surveys.
    \item[] Output: The past survey is deleted.
\end{itemize}

\hfill \break
\textbf{Respondent:}
\begin{itemize}
    \item[] Input: Respondent scans the QR code.
    \item[] Output: Respondent is taken to the correct survey.
    \item[] Input: Respondent completes the survey and clicks submit.
    \item[] Output: Respondent is taken to the thank you page.
\end{itemize}

\hfill \break
\newpage

\subsection{Robustness testing using four test cases (screenshots of both inputs and outputs).}
\newline
\hfill

\textbf{Feedback gatherer:}

\begin{itemize}
    \item[] Input: Feedback gatherer clicks the “Create Survey” button without entering anything.
\item[] Output: Feedback gatherer is shown a warning to select a type of answer and to enter a question.
\item[] Input: Feedback gatherer clicks the “Create Survey” button without entering a question after selecting a type of answer.
\item[] Output: Feedback gatherer is shown a warning to enter a question.
\item[] Input: Feedback gatherer attempts to close the browser or to return to the main page while creating a survey.
\item[] Output: Feedback gatherer is shown a warning that the entered information will be lost.
\end{itemize}
\newline 
\hfill \break

\textbf{Respondent:}
\begin{itemize}
    \item[] Input: Respondent clicks the submit button without answering all of the questions.

Output: Respondent is shown a warning to answer all of the questions.

Input: Respondent attempts to leave the survey without submitting.

Output: Respondent is shown a warning that the entered information will be lost.

\end{itemize}
\newpage 

\subsection{Time-efficiency testing of two functions (with screenshots). Indicate the method you used to measure the time.}
\newline
\hfill \break

\textbf{Feedback gatherer:}
\begin{itemize}
    \item[] Create survey: Time taken for the survey data to be stored in the database and the QR code to be generated.

View responses: Time taken for the response data to be shown to the feedback gatherer after clicking a past survey. 
\end{itemize}
\newline
\hfill \break


\textbf{Respondent:}
\begin{itemize}
    \item[] Go to survey: Time taken for the survey to be loaded from the database when the QR code is scanned.

Submit responses: Time taken for the response data to be stored in the database and to show the respondent that the responses were successfully submitted.
\end{itemize}



\addtolength{\textheight}{-12cm}


%%%%%%%%%%%%%%%%%%%%%%%%%%%%%%%%%%%%%%%%%%%%%%%%%%%%%%%%%%%%%%%%%%%%%%%%%%%%%%%%



%%%%%%%%%%%%%%%%%%%%%%%%%%%%%%%%%%%%%%%%%%%%%%%%%%%%%%%%%%%%%%%%%%%%%%%%%%%%%%%%


%%%%%%%%%%%%%%%%%%%%%%%%%%%%%%%%%%%%%%%%%%%%%%%%%%%%%%%%%%%%%%%%%%%%%%%%%%%%%%%%

\newpage


%\begin{thebibliography}{99}

%\bibitem{young-1964} G. O. Young, Synthetic structure of industrial plastics (Book style with paper title and editor), 	in Plastics, 2nd ed. vol. 3, J. Peters, Ed.  New York: McGraw-Hill, 1964, pp. 15Ð64.
%\bibitem{chen-1993} W.-K. Chen, Linear Networks and Systems (Book style).	Belmont, CA: Wadsworth, 1993, pp. 123Ð135.
%\bibitem{poor-1985} H. Poor, An Introduction to Signal Detection and Estimation.   New York: Springer-Verlag, 1985, ch. 4.

%\bibitem{uofr} "Home", University of Regina, 2022. [Online]. Available: https://www.uregina.ca/. [Accessed: 06- Apr- 2022].

%\end{thebibliography}




\end{document}
