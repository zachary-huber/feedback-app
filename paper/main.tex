\documentclass[letterpaper, 10 pt, conference]{ieeeconf} 
\IEEEoverridecommandlockouts                              
\overrideIEEEmargins




% The following packages can be found on http:\\www.ctan.org
\usepackage{graphics} % for pdf, bitmapped graphics files
%\usepackage{epsfig} % for postscript graphics files
%\usepackage{mathptmx} % assumes new font selection scheme installed
%\usepackage{times} % assumes new font selection scheme installed
%\usepackage{amsmath} % assumes amsmath package installed
%\usepackage{amssymb}  % assumes amsmath package installed
\usepackage{graphicx}

\begin{document}

\thispagestyle{empty}
\pagestyle{empty}

\begin{titlepage}
    \begin{center}
        \vspace*{2.5cm}
 
        \textbf{Feedback Dashboarding Tool}
 
        \vspace{0.5cm}
         Thesis Subtitle
 
         \vfill
        
 
        \textbf{Genevieve Couture}
        200426387 grc839@uregina.ca
 
        \textbf{Payton Gilbertson}
        200248472 gilbertp@uregina.ca
 
        \textbf{Liam Gulak}
        200393578 lgq292@uregina.ca
 
        \textbf{Yunseok Kim}
        200434122 yjk172@uregina.ca
 
        \textbf{Zachary Huber}
        200408160 zeh434@uregina.ca
 
        \vspace{1.5cm}  
        CS 476 Project Report
             
        \vspace{0.8cm}
      
             
        University of Regina\\
        March 21, 2023
             
    \end{center}
 \end{titlepage}
%%%%%%%%%%%%%%%%%%% ABSTRACT %%%%%%%%%%%%%%%%%%%%%%%%%%%%%%%%%%%%%%%%%%%%%%%%%%%%
\begin{abstract}
\end{abstract}

%%%%%%%%%%%%%%%%%%% BODY %%%%%%%%%%%%%%%%%%%%%%%%%%%%%%%%%%%%%%%%%%%%%%%%%%%%%%%%

\section{Problem Definition}
Develop a software that facilitates administration of surveys and feedback collection. The software must be fast, robust, and friendly.

Marketing/Business Operations 

Businesses want meaningful direct feedback on their products and services. Our project aims to provide a generalized solution to context-specific customer feedback monitoring without the need of onsite hardware to capture responses. In other terms, our app is a solution for conducting self-administered surveys.

Our application is a simple tool for conducting self-administered surveys. Our users can log into their profiles to view customer feedback to improve their businesses. QR codes can be generated for wide distribution and quick access. SurveyMonkey (https://www.surveymonkey.com/) and QuestionPro (https://www.questionpro.com/) are comparable existing systems. Some benefits of our application are that the users can received unlimited number of responses for their surveys without cost and all features to create the surveys are free as well. Our application does not have many templates and complicated tools that may negatively impact loading time. Respondents do not need to create an account to respond to surveys. 


    

\section{Application Benefits}

\section{Requirements Elicitation and Specification}
Two types of users/roles: Feedback Gatherer / Respondent.

\textbf{Feedback Gatherer}
    
    Register: Feedback gatherer creates an account by entering the required information
    
    Create Survey: Feedback gatherer creates a survey by entering questions and selecting type of responses
    
    View QR Code: Feedback gatherer views the generated QR code for the survey
    
    Delete Survey: Feedback gatherer deletes a survey that is no longer required
    
    View Statistics and responses: Feedback gatherer views statistics and responses of a selected survey
    
\textbf{Respondent}
    
    Scan QR Code: Respondent scans the QR code to view the survey
    
    Enter responses: Respondent enters responses to the survey questions
    
    Submit: Respondent submits responses to the survey questions

The feedback gatherer can create a survey after registering. The feedback gatherer has the option to delete an existing survey or view statistics and responses to an existing survey. After creating a survey, the feedback gatherer can view the QR code to administer the created survey.

The respondent can scan the QR code to take a survey created by a feedback gatherer. The respondent enters responses to the survey and submit the responses.

\section{Top-level and Low-level Software Design}

\section{Software Construction}

\section{Technical Documentation}

\section{Acceptance Testing}

\section{Conclusion}


\addtolength{\textheight}{-12cm}


%%%%%%%%%%%%%%%%%%%%%%%%%%%%%%%%%%%%%%%%%%%%%%%%%%%%%%%%%%%%%%%%%%%%%%%%%%%%%%%%



%%%%%%%%%%%%%%%%%%%%%%%%%%%%%%%%%%%%%%%%%%%%%%%%%%%%%%%%%%%%%%%%%%%%%%%%%%%%%%%%



%%%%%%%%%%%%%%%%%%%%%%%%%%%%%%%%%%%%%%%%%%%%%%%%%%%%%%%%%%%%%%%%%%%%%%%%%%%%%%%%
\section*{Appendix}

Appendixes should appear before the acknowledgment.


%%%%%%%%%%%%%%%%%%%%%%%%%%%%%%%%%%%%%%%%%%%%%%%%%%%%%%%%%%%%%%%%%%%%%%%%%%%%%%%%

References are important to the reader; therefore, each citation must be complete and correct. If at all possible, references should be commonly available publications.



\begin{thebibliography}{99}

\bibitem{young-1964} G. O. Young, Synthetic structure of industrial plastics (Book style with paper title and editor), 	in Plastics, 2nd ed. vol. 3, J. Peters, Ed.  New York: McGraw-Hill, 1964, pp. 15Ð64.
\bibitem{chen-1993} W.-K. Chen, Linear Networks and Systems (Book style).	Belmont, CA: Wadsworth, 1993, pp. 123Ð135.
\bibitem{poor-1985} H. Poor, An Introduction to Signal Detection and Estimation.   New York: Springer-Verlag, 1985, ch. 4.

\bibitem{uofr} "Home", University of Regina, 2022. [Online]. Available: https://www.uregina.ca/. [Accessed: 06- Apr- 2022].

\end{thebibliography}




\end{document}
